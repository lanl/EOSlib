%%% use twocolumn and 10pt options with the asme2ej format
\documentclass[twocolumn,10pt]{asme2ej}

\usepackage{epsfig} %% for loading postscript figures
\usepackage{amsmath}
\usepackage{nth}    %% simplified ordinate numbers (1st, 2nd, etc.)

%% The class has several options
%  onecolumn/twocolumn - format for one or two columns per page
%  10pt/11pt/12pt - use 10, 11, or 12 point font
%  oneside/twoside - format for oneside/twosided printing
%  final/draft - format for final/draft copy
%  cleanfoot - take out copyright info in footer leave page number
%  cleanhead - take out the conference banner on the title page
%  titlepage/notitlepage - put in titlepage or leave out titlepage
%  
%% The default is oneside, onecolumn, 10pt, final


\title{EOSlib: A reference implementation for thermodynamic equations of state
}

%%% first author
\author{C. Nathan Woods
    \affiliation{
	Postdoctoral Researcher,\\
	Verification and Analysis Group\\
	Los Alamos National Laboratory\\
	Los Alamos, NM 87545\\
    Email: woodscn@lanl.gov
    }	
}

%%% second author
%%% remove the following entry for single author papers
%%% add more entries for additional authors
\author{Ralph Menikoff \\
}

\begin{document}

\maketitle    

%%%%%%%%%%%%%%%%%%%%%%%%%%%%%%%%%%%%%%%%%%%%%%%%%%%%%%%%%%%%%%%%%%%%%%
\begin{abstract}
{\it 
Verification of computational physics codes, whether using test problems or manufactured solutions, necessarily requires the use of an appropriate thermodynamic model. These models are themselves complex and can be challenging to implement. These challenges have greatly impeded the development of verification test problems using complex thermodynamic models, which makes it difficult to establish confidence in physics codes that implement these models. We have developed EOSlib, a stand-alone software package that includes a variety of thermodynamic models. This software library may be used directly or as a reference benchmark against which other implementations of thermodynamic models may be compared. We describe the overall structure and design of the code, and highlight several useful features, including a unified model API, segregated data files, and simple extensibility. We demonstrate these capabilities with a number of examples and discuss plans for incorporating this functionality into ExactPack and the LANL Verification Test Suite.
}
\end{abstract}

% %%%%%%%%%%%%%%%%%%%%%%%%%%%%%%%%%%%%%%%%%%%%%%%%%%%%%%%%%%%%%%%%%%%%%%
% \begin{nomenclature}
% \entry{A}{You may include nomenclature here.}
% \entry{$\alpha$}{There are two arguments for each entry of the nomemclature environment, the symbol and the definition.}
% \end{nomenclature}

%%%%%%%%%%%%%%%%%%%%%%%%%%%%%%%%%%%%%%%%%%%%%%%%%%%%%%%%%%%%%%%%%%%%%%
\section{Introduction}

Equilibrium thermodynamics, which describes the macroscopic behavior of large systems of interacting particles, is a fundamental science for fluid dynamics, materials modeling, astrophysics, and overall continuum mechanics. Unfortunately, despite the fact that the mathematics have been essentially known since the 19th century, actually performing thermodynamic calculations remains a tedious, involved process for most models. 

The fundamental operation of a thermodynamics model is typically quite straight-forward: given some set of thermodynamic state variables, compute the remaining variables corresponding to that state, consistent with both conservation of energy and maximization of entropy. Different thermodynamic models change the details of these computations, but the overall structure is identical for a wide variety of systems. This invariance in the computation immediately suggests the possibility of automating thermodynamic calculations using an object-oriented software library. We have created EOSlib, a library that provides consistent access to a reference implementation of a variety of equation-of-state (EOS) models, a database of model parameters, and tools to automate purely thermodynamic calculations for a variety of experimentally useful but mathematically non-trivial quantities. We are making EOSlib publicly available in order to enable scientists to: use the EOS models provided to benchmark the accuracy of other, more performant implementations; easily compare different EOS models within a common interface; access, use, and contribute to high-quality EOS parameters with a known and documented pedigree; easily answer thermodynamic questions about a given material that may otherwise require involved computations.

\section{Theoretical Thermodynamics}
In order to describe the use and utility of EOSlib, it is necessary to briefly review the relevant thermodynamic theory. This review is only cursory, and the reader is referred to the cited sources for more detailed information.

Thermodynamics is fundamentally a science that describes the macroscopic behavior of a system composed of a large number of interacting particles.\cite{RN1010} Simple systems may be described completely by their internal energy $U$, volume $V$, and the number of particles $N$, and the particular state of a system will maximize the entropy of the system $S\left(U,V,N\right)$. Alternatively (and more commonly), the entropy function may be inverted with respect to energy, such that the entropy and volume assume the values that minimize the internal energy $U\left(S,V,N\right)$. These functions contain all of the information required to completely describe a given system, and are known as fundamental relations, or also thermodynamic potentials. 

\subsection{Derived Quantities}

\subsubsection{\nth{1} derivatives (Equations of State)}
The first differential of the internal energy may be written:
% MathType!MTEF!2!1!+-
% faaagCart1ev2aaaKnaaaaWenf2ys9wBH5garuWqH1MyYLwyaeXatL
% xBI9gBaerbd9wDYLwzYbqee0evGueE0jxyaibaieYlf9irVeeu0dXd
% h9vqqj-hHeeu0xXdbba9frFj0-OqFfea0dXdd9vqaq-JfrVkFHe9pg
% ea0dXdar-Jb9hs0dXdbPYxe9vr0-vr0-vqpi0dc9GqpWqaaeaabiGa
% ciaacaqabeaadaqaaqaaaOqaaiaadsgacaWGvbGaeyypa0Jaamivai
% aadsgacaWGtbGaeyOeI0IaamiuaiaadsgacaWGwbGaey4kaSIaeqiV
% d0Maamizaiaad6eaaaa!3C03!
\[dU = TdS - PdV + \mu dN\]
where % MathType!MTEF!2!1!+-
% faaagCart1ev2aqaKnaaaaWenf2ys9wBH5garuWqH1MyYLwyaeXatL
% xBI9gBaerbd9wDYLwzYbqee0evGueE0jxyaibaieYlf9irVeeu0dXd
% h9vqqj-hHeeu0xXdbba9frFj0-OqFfea0dXdd9vqaq-JfrVkFHe9pg
% ea0dXdar-Jb9hs0dXdbPYxe9vr0-vr0-vqpi0dc9GqpWqaaeaabiGa
% ciaacaqabeaadaqaaqaaaOqaaiaadsfacqGHHjIUdaqadaqaamaala
% aabaGaeyOaIyRaamyvaaqaaiabgkGi2kaadofaaaaacaGLOaGaayzk
% aaWaaSbaaSqaaiaadAfacaGGSaGaamOtaaqabaaaaa!3A0A!
$T \equiv {\left( {\frac{{\partial U}}{{\partial S}}} \right)_{V,N}}$, 
and % MathType!MTEF!2!1!+-
% faaagCart1ev2aqaKnaaaaWenf2ys9wBH5garuWqH1MyYLwyaeXatL
% xBI9gBaerbd9wDYLwzYbqee0evGueE0jxyaibaieYlf9irVeeu0dXd
% h9vqqj-hHeeu0xXdbba9frFj0-OqFfea0dXdd9vqaq-JfrVkFHe9pg
% ea0dXdar-Jb9hs0dXdbPYxe9vr0-vr0-vqpi0dc9GqpWqaaeaabiGa
% ciaacaqabeaadaqaaqaaaOqaaiaadcfacqGHHjIUcqGHsisldaqada
% qaamaalaaabaGaeyOaIyRaamyvaaqaaiabgkGi2kaadAfaaaaacaGL
% OaGaayzkaaWaaSbaaSqaaiaadofacaGGSaGaamOtaaqabaaaaa!3AF3!
$P \equiv  - {\left( {\frac{{\partial U}}{{\partial V}}} \right)_{S,N}}$
correspond to the usual definitions of temperature and pressure, and 
% MathType!MTEF!2!1!+-
% faaagCart1ev2aqaKnaaaaWenf2ys9wBH5garuWqH1MyYLwyaeXatL
% xBI9gBaerbd9wDYLwzYbqee0evGueE0jxyaibaieYlf9irVeeu0dXd
% h9vqqj-hHeeu0xXdbba9frFj0-OqFfea0dXdd9vqaq-JfrVkFHe9pg
% ea0dXdar-Jb9hs0dXdbPYxe9vr0-vr0-vqpi0dc9GqpWqaaeaabiGa
% ciaacaqabeaadaqaaqaaaOqaaiabeY7aTjabggMi6oaabmaabaWaaS
% aaaeaacqGHciITcaWGvbaabaGaeyOaIyRaamOtaaaaaiaawIcacaGL
% PaaadaWgaaWcbaGaam4uaiaacYcacaWGwbaabeaaaaa!3AE7!
$\mu  \equiv {\left( {\frac{{\partial U}}{{\partial N}}} \right)_{S,V}}$
is the chemical potential. These first derivatives of the energy function are called equations of state. 
Given the properties of fundamental relation, it is possible to derive two equations relating these equations of state. The first is the Euler relation:
% MathType!MTEF!2!1!+-
% faaagCart1ev2aqaKnaaaaWenf2ys9wBH5garuWqH1MyYLwyaeXatL
% xBI9gBaerbd9wDYLwzYbqee0evGueE0jxyaibaieYlf9irVeeu0dXd
% h9vqqj-hHeeu0xXdbba9frFj0-OqFfea0dXdd9vqaq-JfrVkFHe9pg
% ea0dXdar-Jb9hs0dXdbPYxe9vr0-vr0-vqpi0dc9GqpWqaaeaabiGa
% ciaacaqabeaadaqaaqaaaOqaaiaadwfacqGH9aqpcaWGubGaaGPaVl
% aadofacqGHsislcaWGqbGaaGPaVlaadAfacqGHRaWkcqaH8oqBcaaM
% c8UaamOtaaaa!3D01!
\[U = T\,S - P\,V + \mu \,N\].
The second is the Gibbs-Duhem relation:
% MathType!MTEF!2!1!+-
% faaagCart1ev2aaaKnaaaaWenf2ys9wBH5garuWqH1MyYLwyaeXatL
% xBI9gBaerbd9wDYLwzYbqee0evGueE0jxyaibaieYlf9irVeeu0dXd
% h9vqqj-hHeeu0xXdbba9frFj0-OqFfea0dXdd9vqaq-JfrVkFHe9pg
% ea0dXdar-Jb9hs0dXdbPYxe9vr0-vr0-vqpi0dc9GqpWqaaeaabiGa
% ciaacaqabeaadaqaaqaaaOqaaiaaicdacqGH9aqpcaWGtbGaaGPaVl
% aadsgacaWGubGaeyOeI0IaamOvaiaaykW7caWGKbGaamiuaiabgUca
% Riaad6eacaaMc8UaamizaiabeY7aTbaa!3F9B!
\[0 = S\,dT - V\,dP + N\,d\mu \].

Using these two equations, it is possible to fully define a fundamental relation given only two of three equations of state, to within an integration constant. 

It is also common to work in terms of molar (or, equivalently, mass-specific) quantities: $s$, $v$, $u$. In this representation, we have: 
% MathType!MTEF!2!1!+-
% faaagCart1ev2aaaKnaaaaWenf2ys9wBH5garuWqH1MyYLwyaeXatL
% xBI9gBaerbd9wDYLwzYbqee0evGueE0jxyaibaieYlf9irVeeu0dXd
% h9vqqj-hHeeu0xXdbba9frFj0-OqFfea0dXdd9vqaq-JfrVkFHe9pg
% ea0dXdar-Jb9hs0dXdbPYxe9vr0-vr0-vqpi0dc9GqpWqaaeaabiGa
% ciaacaqabeaadaqaaqaaaOabaiqabaGaamyDamaabmaabaGaam4Cai
% aacYcacaWG2baacaGLOaGaayzkaaaabaGaamivamaabmaabaGaam4C
% aiaacYcacaWG2baacaGLOaGaayzkaaaabaGaamiuamaabmaabaGaam
% 4CaiaacYcacaWG2baacaGLOaGaayzkaaaabaGaamizaiaadwhacqGH
% 9aqpcaWGubGaaGPaVlaadsgacaWGZbGaeyOeI0IaamiuaiaaykW7ca
% WGKbGaamODaaaaaa!4A5B!
\[\begin{array}{c}
u\left( {s,v} \right)\\
T\left( {s,v} \right)\\
P\left( {s,v} \right)\\
du = T\,ds - P\,dv
\end{array}\]

Again, given two equations of state, these may be integrated to yield the fundamental relation to within an integration constant.

\subsubsection{\nth{2} derivatives}
Just as the first derivatives of the energy correspond to physically relevant quantities, many of the second derivatives are likewise familiar. For simple systems in terms of specific quantities, some commonly measured quantities are:
% MathType!MTEF!2!1!+-
% faaagCart1ev2aaaKnaaaaWenf2ys9wBH5garuWqH1MyYLwyaeXatL
% xBI9gBaerbd9wDYLwzYbqee0evGueE0jxyaibaieYlf9irVeeu0dXd
% h9vqqj-hHeeu0xXdbba9frFj0-OqFfea0dXdd9vqaq-JfrVkFHe9pg
% ea0dXdar-Jb9hs0dXdbPYxe9vr0-vr0-vqpi0dc9GqpWqaaeaabiGa
% ciaacaqabeaadaqaaqaaaOqaauaabeqafiaaaaqaaiaadoeadaWgaa
% WcbaGaamODaaqabaGccqGH9aqpcaWGubWaaeWaaeaadaWcaaqaaiab
% gkGi2kaadohaaeaacqGHciITcaWGubaaaaGaayjkaiaawMcaamaaBa
% aaleaacaWG2baabeaaaOqaaiaabofacaqGWbGaaeyzaiaabogacaqG
% PbGaaeOzaiaabMgacaqGJbGaaeiiaiaabIgacaqGLbGaaeyyaiaabs
% hacaqGGaGaaeyyaiaabshacaqGGaGaae4yaiaab+gacaqGUbGaae4C
% aiaabshacaqGHbGaaeOBaiaabshacaqGGaGaaeODaiaab+gacaqGSb
% GaaeyDaiaab2gacaqGLbaabaGaam4qamaaBaaaleaacaWGqbaabeaa
% kiabg2da9iaadsfadaqadaqaamaalaaabaGaeyOaIyRaam4Caaqaai
% abgkGi2kaadsfaaaaacaGLOaGaayzkaaWaaSbaaSqaaiaadcfaaeqa
% aaGcbaGaae4uaiaabchacaqGLbGaae4yaiaabMgacaqGMbGaaeyAai
% aabogacaqGGaGaaeiAaiaabwgacaqGHbGaaeiDaiaabccacaqGHbGa
% aeiDaiaabccacaqGJbGaae4Baiaab6gacaqGZbGaaeiDaiaabggaca
% qGUbGaaeiDaiaabccacaqGWbGaaeOCaiaabwgacaqGZbGaae4Caiaa
% bwhacaqGYbGaaeyzaaqaaiaadUeadaWgaaWcbaGaamivaaqabaGccq
% GH9aqpcqGHsisldaWcaaqaaiaaigdaaeaacaWG2baaamaabmaabaWa
% aSaaaeaacqGHciITcaWG2baabaGaeyOaIyRaamiuaaaaaiaawIcaca
% GLPaaadaWgaaWcbaGaamivaaqabaaakeaacaqGjbGaae4Caiaab+ga
% caqG0bGaaeiAaiaabwgacaqGYbGaaeyBaiaabggacaqGSbGaaeiiai
% aabogacaqGVbGaaeyBaiaabchacaqGYbGaaeyzaiaabohacaqGZbGa
% aeyAaiaabkgacaqGPbGaaeiBaiaabMgacaqG0bGaaeyEaaqaaiaadU
% eadaWgaaWcbaGaam4CaaqabaGccqGH9aqpcqGHsisldaWcaaqaaiaa
% igdaaeaacaWG2baaamaabmaabaWaaSaaaeaacqGHciITcaWG2baaba
% GaeyOaIyRaamiuaaaaaiaawIcacaGLPaaadaWgaaWcbaGaam4Caaqa
% baaakeaacaqGjbGaae4CaiaabwgacaqGUbGaaeiDaiaabkhacaqGVb
% GaaeiCaiaabMgacaqGJbGaaeiiaiaabogacaqGVbGaaeyBaiaabcha
% caqGYbGaaeyzaiaabohacaqGZbGaaeyAaiaabkgacaqGPbGaaeiBai
% aabMgacaqG0bGaaeyEaaqaaiabek7aIjabg2da9maalaaabaGaaGym
% aaqaaiaadAhaaaWaaeWaaeaadaWcaaqaaiabgkGi2kaadAhaaeaacq
% GHciITcaWGubaaaaGaayjkaiaawMcaamaaBaaaleaacaWGqbaabeaa
% aOqaaiaaboeacaqGVbGaaeyzaiaabAgacaqGMbGaaeyAaiaabogaca
% qGPbGaaeyzaiaab6gacaqG0bGaaeiiaiaab+gacaqGMbGaaeiiaiaa
% bshacaqGObGaaeyzaiaabkhacaqGTbGaaeyyaiaabYgacaqGGaGaae
% yzaiaabIhacaqGWbGaaeyyaiaab6gacaqGZbGaaeyAaiaab+gacaqG
% Ubaaaaaa!F22B!
\[\begin{array}{*{20}{c}}
{{C_v} = T{{\left( {\frac{{\partial s}}{{\partial T}}} \right)}_v}}&{{\text{Specific heat at constant volume}}}\\
{{C_P} = T{{\left( {\frac{{\partial s}}{{\partial T}}} \right)}_P}}&{{\text{Specific heat at constant pressure}}}\\
{{K_T} =  - \frac{1}{v}{{\left( {\frac{{\partial v}}{{\partial P}}} \right)}_T}}&{{\text{Isothermal compressibility}}}\\
{{K_s} =  - \frac{1}{v}{{\left( {\frac{{\partial v}}{{\partial P}}} \right)}_s}}&{{\text{Isentropic compressibility}}}\\
{\beta  = \frac{1}{v}{{\left( {\frac{{\partial v}}{{\partial T}}} \right)}_P}}&{{\text{Coefficient of thermal expansion}}}
\end{array}\]

However, it is important to recognize that the second derivatives of a thermodynamic potential are not all independent. For instance:

% MathType!MTEF!2!1!+-
% faaagCart1ev2aaaKnaaaaWenf2ys9wBH5garuWqH1MyYLwyaeXatL
% xBI9gBaerbd9wDYLwzYbqee0evGueE0jxyaibaieYlf9irVeeu0dXd
% h9vqqj-hHeeu0xXdbba9frFj0-OqFfea0dXdd9vqaq-JfrVkFHe9pg
% ea0dXdar-Jb9hs0dXdbPYxe9vr0-vr0-vqpi0dc9GqpWqaaeaabiGa
% ciaacaqabeaadaqaaqaaaOabaiqabaGaamyDamaabmaabaGaam4Cai
% aacYcacaWG2baacaGLOaGaayzkaaaabaGaey40H8nabaqbaeqabeGa
% aaqaamaabmaabaWaaSaaaeaacqGHciITcaWG1baabaGaeyOaIyRaam
% 4CaaaaaiaawIcacaGLPaaadaWgaaWcbaGaamODaaqabaaakeaadaqa
% daqaamaalaaabaGaeyOaIyRaamyDaaqaaiabgkGi2kaadAhaaaaaca
% GLOaGaayzkaaWaaSbaaSqaaiaadohaaeqaaaaaaOqaaiabgoDiFdqa
% auaabeqabmaaaeaadaqadaqaamaalaaabaGaeyOaIylabaGaeyOaIy
% RaamODaaaadaqadaqaamaalaaabaGaeyOaIyRaamyDaaqaaiabgkGi
% 2kaadohaaaaacaGLOaGaayzkaaWaaSbaaSqaaiaadAhaaeqaaaGcca
% GLOaGaayzkaaWaaSbaaSqaaiaadohaaeqaaaGcbaGaeyypa0dabaWa
% aeWaaeaadaWcaaqaaiabgkGi2cqaaiabgkGi2kaadohaaaWaaeWaae
% aadaWcaaqaaiabgkGi2kaadwhaaeaacqGHciITcaWG2baaaaGaayjk
% aiaawMcaamaaBaaaleaacaWGZbaabeaaaOGaayjkaiaawMcaamaaBa
% aaleaacaWG2baabeaaaaaaaaa!6509!
\[\begin{array}{c}
u\left( {s,v} \right)\\
 \Downarrow \\
\begin{array}{*{20}{c}}
{{{\left( {\frac{{\partial u}}{{\partial s}}} \right)}_v}}&{{{\left( {\frac{{\partial u}}{{\partial v}}} \right)}_s}}
\end{array}\\
 \Downarrow \\
\begin{array}{*{20}{c}}
{{{\left( {\frac{\partial }{{\partial v}}{{\left( {\frac{{\partial u}}{{\partial s}}} \right)}_v}} \right)}_s}}& = &{{{\left( {\frac{\partial }{{\partial s}}{{\left( {\frac{{\partial u}}{{\partial v}}} \right)}_s}} \right)}_v}}
\end{array}
\end{array}\]

If we assume that mixed partial derivatives commute, then the equality is obvious. It becomes less obvious when we write these in terms of temperature and pressure, which yields the equality:

% MathType!MTEF!2!1!+-
% faaagCart1ev2aaaKnaaaaWenf2ys9wBH5garuWqH1MyYLwyaeXatL
% xBI9gBaerbd9wDYLwzYbqee0evGueE0jxyaibaieYlf9irVeeu0dXd
% h9vqqj-hHeeu0xXdbba9frFj0-OqFfea0dXdd9vqaq-JfrVkFHe9pg
% ea0dXdar-Jb9hs0dXdbPYxe9vr0-vr0-vqpi0dc9GqpWqaaeaabiGa
% ciaacaqabeaadaqaaqaaaOqaamaabmaabaWaaSaaaeaacqGHciITca
% WGubaabaGaeyOaIyRaamODaaaaaiaawIcacaGLPaaadaWgaaWcbaGa
% am4CaaqabaGccqGH9aqpcqGHsisldaqadaqaamaalaaabaGaeyOaIy
% RaamiuaaqaaiabgkGi2kaadohaaaaacaGLOaGaayzkaaWaaSbaaSqa
% aiaadAhaaeqaaaaa!3F79!
\[{\left( {\frac{{\partial T}}{{\partial v}}} \right)_s} =  - {\left( {\frac{{\partial P}}{{\partial s}}} \right)_v}\]

There are many such relations for different thermodynamic potentials, all stemming from the commutivity of mixed partial derivatives. They are collectively known as Maxwell relations. 

For simple thermodynamic systems, there are only three independent second derivatives. Many different sets are used\cite{RN1010,RN819}, depending on the intended application. As an example, for applications in shock and detonation physics, a convenient set of second derivatives is:

% MathType!MTEF!2!1!+-
% faaagCart1ev2aaaKnaaaaWenf2ys9wBH5garuWqH1MyYLwyaeXatL
% xBI9gBaerbd9wDYLwzYbqee0evGueE0jxyaibaieYlf9irVeeu0dXd
% h9vqqj-hHeeu0xXdbba9frFj0-OqFfea0dXdd9vqaq-JfrVkFHe9pg
% ea0dXdar-Jb9hs0dXdbPYxe9vr0-vr0-vqpi0dc9GqpWqaaeaabiGa
% ciaacaqabeaadaqaaqaaaOqaauaabeqadiaaaeaacqaHZoWzcqGHHj
% IUdaWcaaqaaiaaigdaaeaacaWGqbGaam4samaaBaaaleaacaWGZbaa
% beaaaaaakeaacaqGbbGaaeizaiaabMgacaqGHbGaaeOyaiaabggaca
% qG0bGaaeyAaiaabogacaqGGaGaaeyzaiaabIhacaqGWbGaae4Baiaa
% b6gacaqGLbGaaeOBaiaabshaaeaacqqHtoWrcqGHHjIUdaWcaaqaai
% abek7aIjaadAhaaeaacaWGdbWaaSbaaSqaaiaadAhaaeqaaOGaam4s
% amaaBaaaleaacaWGubaabeaaaaaakeaacaqGhbGaaeOCaiqabwhaga
% Waaiaab6gacaqGLbGaaeyAaiaabohacaqGLbGaaeOBaiaabccacaqG
% JbGaae4BaiaabwgacaqGMbGaaeOzaiaabMgacaqGJbGaaeyAaiaabw
% gacaqGUbGaaeiDaaqaaiaadEgacqGHHjIUdaWcaaqaaiaadcfacaWG
% 2baabaGaam4qamaaBaaaleaacaWG2baabeaakiaadsfaaaaabaGaae
% iraiaabMgacaqGTbGaaeyzaiaab6gacaqGZbGaaeyAaiaab+gacaqG
% UbGaaeiBaiaabwgacaqGZbGaae4CaiaabccacaqGZbGaaeiCaiaabw
% gacaqGJbGaaeyAaiaabAgacaqGPbGaae4yaiaabccacaqGObGaaeyz
% aiaabggacaqG0baaaaaa!82AA!
\[\begin{array}{*{20}{c}}
{\gamma  \equiv \frac{1}{{P{K_s}}}}&{{\text{Adiabatic exponent}}}\\
{\Gamma  \equiv \frac{{\beta v}}{{{C_v}{K_T}}}}&{{\text{Gr\"uneisen coefficient}}}\\
{g \equiv \frac{{Pv}}{{{C_v}T}}}&{{\text{Dimensionless specific heat}}}
\end{array}\]

Menikoff and Plohr\cite{RN819} give a useful and exhaustive list of thermodynamic identies in terms of this set. 

\subsubsection{\nth{3} derivatives}
There are many possible \nth{3} derivatives. One that is of particular interest is known as the fundamental derivative, given by\cite{RN819}:

% MathType!MTEF!2!1!+-
% faaagCart1ev2aaaKnaaaaWenf2ys9wBH5garuWqH1MyYLwyaeXatL
% xBI9gBaerbd9wDYLwzYbqee0evGueE0jxyaibaieYlf9irVeeu0dXd
% h9vqqj-hHeeu0xXdbba9frFj0-OqFfea0dXdd9vqaq-JfrVkFHe9pg
% ea0dXdar-Jb9hs0dXdbPYxe9vr0-vr0-vqpi0dc9GqpWqaaeaabiGa
% ciaacaqabeaadaqaaqaaaOqaamrr1ngBPrwtHrhAXaqeguuDJXwAKb
% stHrhAG8KBLbacfaGae8NbXFKaeyyyIO7aaSaaaeaacaaIXaaabaGa
% aGOmaaaacaWG2bWaaSaaaeaadaabcaqaamaalyaabaGaeyOaIy7aaW
% baaSqabeaacaaIZaaaaOGaamyDaaqaaiabgkGi2kaadAhadaahaaWc
% beqaaiaaiodaaaaaaaGccaGLiWoadaWgaaWcbaGaam4Caaqabaaake
% aadaabcaqaamaalyaabaGaeyOaIy7aaWbaaSqabeaacaaIYaaaaOGa
% amyDaaqaaiabgkGi2kaadAhadaahaaWcbeqaaiaaikdaaaaaaaGcca
% GLiWoadaWgaaWcbaGaam4CaaqabaaaaOGaeyypa0ZaaSaaaeaacaaI
% XaaabaGaaGOmaaaadaqadaqaaiabeo7aNjabgUcaRiaaigdacqGHsi
% sldaWcaaqaaiaadAhaaeaacqaHZoWzaaWaaeWaaeaadaWcaaqaaiab
% gkGi2kabeo7aNbqaaiabgkGi2kaadAhaaaaacaGLOaGaayzkaaWaaS
% baaSqaaiaadohaaeqaaaGccaGLOaGaayzkaaaaaa!64A0!
\[{\cal G} \equiv \frac{1}{2}v\frac{{{{\left. {{{{\partial ^3}u} \mathord{\left/
 {\vphantom {{{\partial ^3}u} {\partial {v^3}}}} \right.
 \kern-\nulldelimiterspace} {\partial {v^3}}}} \right|}_s}}}{{{{\left. {{{{\partial ^2}u} \mathord{\left/
 {\vphantom {{{\partial ^2}u} {\partial {v^2}}}} \right.
 \kern-\nulldelimiterspace} {\partial {v^2}}}} \right|}_s}}} = \frac{1}{2}\left( {\gamma  + 1 - \frac{v}{\gamma }{{\left( {\frac{{\partial \gamma }}{{\partial v}}} \right)}_s}} \right)\]

This derivative is described by Thompson\cite{RN1017}, and measures the convexity of isentropes in the P-V plane. It is also related to the behavior of shocks in a given material.

\subsubsection{Thermodynamic potentials}
In the standard energy formulation, entropy and volume play the role of independent variables. This is inconvenient, because these variables can be challenging to measure and control. Using the theory of Legendre transformations, it is possible to define alternative fundamental relations such that pressure and temperature become the independent variables, without any loss of information about the system. The most common of these are the enthalpy $H\left(S, P, N\right)$, Helmholtz potential $F\left(T, V, N\right)$, and Gibbs potential $G\left(T, P, N\right)$. Each of these has a corresponding Euler relation, Gibbs-Duhem relation, and set of Maxwell relations. Each of these concepts also has an equivalent representation in the entropy formulation, but these will not be discussed here. 

\subsection{Completeness}
A complete equation of state is simply an EOS specification that uniquely defines the thermodynamic state of the system, as described above. This is best understood in contrast with an incomplete EOS, such as one which specifies only the functional form of pressure while ignoring the other variables. For many applications, (including hydrodynamic modeling) an incomplete EOS is sufficient.

\subsection{Summary of thermodynamic constraints}
In summary, we have reviewed how the fundamental relation (or thermodynamic potential) of a system completely specifies its thermodynamic behavior. That is, given the functional form of the potential, one can derive the functional form of every other thermodynamic quantity of interest. Any EOS model that obeys these constraints is said to be thermodynamically consistent. 

We have also reviewed the many ways in which these derived quantities are interrelated, and the constraints that they satisfy:
\begin{itemize}
	\item Euler relation
	\item Gibbs-Duhem relation
	\item Maxwell relations
\end{itemize}
It is important to recognize these constraints, because it is very rare that one is able to obtain the fundamental relation. Instead, a material is typically characterized by direct or indirect measurements of first or second derivatives. Constructing a thermodynamic model that is consistent with both measurements and the theoretical framework described here is a challenging process, and requires many non-trivial thermodynamic computations that can be greatly simplified by a software library such as EOSlib. 

\subsection{Extended EOS}
\begin{itemize}
	\item Includes internal degrees of freedom (i.e. chemical composition)
	\item Includes rate laws
	\item Frozen and equilibrium states?
\end{itemize}


%Note from Ralph: Thermodynamic stability requires Cv >= 0 and Cp >= Cv and Kt >= 0 and Ks >= Kt.

\section{EOSlib}

EOSlib is designed to greatly simplify thermodynamics calculations by bridging the gaps between the theoretical framework and the reality of thermodynamic practice. It provides tools and utilities to compute state changes for general thermodynamic systems, a collection of specific model implementations (such as the ideal gas or JWL thermodynamic models), and database tools for managing the various parameters that describe the behavior of a particular material for a given thermodynamic model. 

 

\subsection{Generic math and program I/O routines - CClib}

\subsubsection{OneDFunction}
Univariate root finder. Brackets the root, then uses bisection and Newton iteration. Used in Hugoniot and in ODE, as well as ImpedanceMatch.

\subsubsection{ODE}
\nth{4}-order Runge-Kutta ODE solver, variable time step (adaptive). Automatically tests accuracy vs. half-size time steps, refines or coarsens as needed. Will use OneDFunction plus interpolation to find the end state (i.e. $t|\lambda(t)=1$). Caches previous values, which prevents the accumulation of round-off errors over many passes back and forth (such as in ImpedanceMatch). 


\subsubsection{Spline}
Used for EOS UpUs (standard for metal). Assumes linear relation between Us, Up. Spline allows you to fit a nonlinear relation to a collection of Us/Up points. C1-continuous, which makes sound speed smooth. Could be deprecated and replaced.

\subsubsection{Calc}
String-based math, used by unit conversions and database readers.

\subsubsection{LinkList}
Just a normal linked-list thing. Used for ODE cache. Could be deprecated and replaced.

\subsubsection{String, IO, Math}
Could be deprecated and replaced.

\subsection{DataBase}


\subsubsection{Database files}
\paragraph{Format}
\paragraph{Loading EOS modules}

\subsubsection{Library functions}
\paragraph{FetchEOS}
\paragraph{Units}
\paragraph{Handle}
\paragraph{Parameters}

\subsection{EOS (abstract base class for EOS model)}
Does not check and enforce thermodynamic consistency, though it will use it to compute things for you. Basically, this means that you need to be careful when specifying $s\left(v,e\right)$ and the \nth{2} derivatives. $P$ and $T$ need to satisfy the Maxwell conditions.

Note: parameterized in terms of $v$, $e$. 

\subsection{The simplest possible EOS}
A subclass MUST provide $P\left(v,e\right)$.

\paragraph{Example: Hugoniot curve}
It is very challenging to obtain direct thermodynamic data for certain thermodynamic regimes. For high-temperatures and pressures, it is much more common to measure hydrodynamic properties, from which thermodynamic state variables may be derived. Shock-Hugoniot experiments use measurements of shock- and particle-velocity in order to determine the change in thermodynamic state variables across a hydrodynamic shock. When parameterized by the shock speed, these form the Hugoniot curve in the thermodynamic variables. The details are as follows:

First, consider a one-dimensional flow with a steady jump discontinuity (shock). Such a flow is described by the conservation equations of mass momentum and energy\cite{RN818}:

% MathType!MTEF!2!1!+-
% faaagCart1ev2aaaKnaaaaWenf2ys9wBH5garuWqH1MyYLwyaeXatL
% xBI9gBaerbd9wDYLwzYbqee0evGueE0jxyaibaieYlf9irVeeu0dXd
% h9vqqj-hHeeu0xXdbba9frFj0-OqFfea0dXdd9vqaq-JfrVkFHe9pg
% ea0dXdar-Jb9hs0dXdbPYxe9vr0-vr0-vqpi0dc9GqpWqaaeaabiGa
% ciaacaqabeaadaqaaqaaaOqaauaabeqadiaaaeaacqaHbpGCdaWgaa
% WcbaGaaGimaaqabaGccaWG1bWaaSbaaSqaaiaadohaaeqaaOGaeyyp
% a0JaeqyWdi3aaeWaaeaacaWG1bWaaSbaaSqaaiaadohaaeqaaOGaey
% OeI0IaamyDaaGaayjkaiaawMcaaaqaaiaacIcacaWGTbGaamyyaiaa
% dohacaWGZbGaaiykaaqaaiaadchacqGHsislcaWGWbWaaSbaaSqaai
% aaicdaaeqaaOGaeyypa0JaeqyWdi3aaSbaaSqaaiaaicdaaeqaaOGa
% amyDaiaadwhadaWgaaWcbaGaam4CaaqabaaakeaacaGGOaGaamyBai
% aad+gacaWGTbGaamyzaiaad6gacaWG0bGaamyDaiaad2gacaGGPaaa
% baGaamyzaiabgUcaRiaadchacaWG2bGaey4kaSYaaSaaaeaacaaIXa
% aabaGaaGOmaaaadaqadaqaaiaadwhadaWgaaWcbaGaam4CaaqabaGc
% cqGHsislcaWG1baacaGLOaGaayzkaaWaaWbaaSqabeaacaaIYaaaaO
% Gaeyypa0JaamyzamaaBaaaleaacaaIWaaabeaakiabgUcaRiaadcha
% daWgaaWcbaGaaGimaaqabaGccaWG2bWaaSbaaSqaaiaaicdaaeqaaO
% Gaey4kaSYaaSaaaeaacaaIXaaabaGaaGOmaaaacaWG1bWaaSbaaSqa
% aiaadohaaeqaaaGcbaGaaiikaiaadwgacaWGUbGaamyzaiaadkhaca
% WGNbGaamyEaiaacMcaaaaaaa!7496!
\[\begin{array}{*{20}{c}}
	{{\rho _0}{u_s} = \rho \left( {{u_s} - u} \right)}&{\text{(mass)}}\\
	{p - {p_0} = {\rho _0}u{u_s}}&{\text{(momentum)}}\\
	{e + pv + \frac{1}{2}{{\left( {{u_s} - u} \right)}^2} = {e_0} + {p_0}{v_0} + \frac{1}{2}{u_s}}&{\text{(energy)}}
\end{array}\]

These equations may be solved together to yield equations the Hugoniot curve, which describes the possible states that result from a thermodynamic system interacting with a shock\cite{RN818}:
% MathType!MTEF!2!1!+-
% faaagCart1ev2aaaKnaaaaWenf2ys9wBH5garuWqH1MyYLwyaeXatL
% xBI9gBaerbd9wDYLwzYbqee0evGueE0jxyaibaieYlf9irVeeu0dXd
% h9vqqj-hHeeu0xXdbba9frFj0-OqFfea0dXdd9vqaq-JfrVkFHe9pg
% ea0dXdar-Jb9hs0dXdbPYxe9vr0-vr0-vqpi0dc9GqpWqaaeaabiGa
% ciaacaqabeaadaqaaqaaaOqaauaabeqabiaaaeaacqqHuoarcaWGLb
% Gaey4kaSIabmiCayaaraGaeuiLdqKaamODaiabg2da9iaaicdaaeaa
% caGGOaGaamisaiaadwhacaWGNbGaam4Baiaad6gacaWGPbGaam4Bai
% aadshacaWGdbGaamyDaiaadkhacaWG2bGaamyzaiaacMcaaaaaaa!449F!
\[\begin{array}{*{20}{c}}
	{\Delta e + \bar p\Delta v = 0}&{\text{(Hugoniot Curve)}}
\end{array}\]

Because the Hugoniot curve is so important in certain regimes, it is important to be able to compute it easily. Given the relation $p\left(v,e\right)$, the Hugoniot can be numerically solved to yield 
% MathType!MTEF!2!1!+-
% faaagCart1ev2aaaKnaaaaWenf2ys9wBH5garuWqH1MyYLwyaeXatL
% xBI9gBaerbd9wDYLwzYbqee0evGueE0jxyaibaieYlf9irVeeu0dXd
% h9vqqj-hHeeu0xXdbba9frFj0-OqFfea0dXdd9vqaq-JfrVkFHe9pg
% ea0dXdar-Jb9hs0dXdbPYxe9vr0-vr0-vqpi0dc9GqpWqaaeaabiGa
% ciaacaqabeaadaqaaqaaaOqaaiaadchadaWgaaWcbaGaamisaaqaba
% GcdaqadaqaaiaadAhaaiaawIcacaGLPaaacqGHHjIUcaWGWbWaaeWa
% aeaacaWG2baacaGLOaGaayzkaaGaam4Caiaac6cacaWG0bGaaiOlai
% abfs5aejaadwgacqGHRaWkceWGWbGbaebadaqadaqaaiaadAhacaGG
% SaGaamyzaaGaayjkaiaawMcaaiabfs5aejaadAhacqGH9aqpcaaIWa
% aaaa!4857!
\[{p_H}\left( v \right) \equiv p\left( v \right)s.t.\Delta e + \bar p\left( {v,e} \right)\Delta v = 0\].

A simple example is given in:


The Hugoniot curve for PMMA at reference state: can be plotted:

\subsubsection{Completing the EOS}
By specifying only the pressure, one obtains an incomplete equation of state. The base class allows the user to complete this EOS by further specifying $s\left(v,e\right)$ and $T\left(v,e\right)$. Care must be taken that these are consistent with $p\left(v,e\right)$ and with each other, especially since $s\left(v,e\right)$ is a fundamental relation.

\paragraph{Example: Isotherm}
A complete EOS completely defines the thermodynamic system, provided that it is thermodynamically consistent. This allows the computation of additional useful relations, such as isotherms that can be compared with experimental cold curves.

\subsubsection{\nth{2} derivatives (sound speed, gruneisen coefficient, specific heat, etc.) can be computed numerically but should be specified directly.}
Note: All other second derivatives can be computed from these.

Why would you specify them?

\paragraph{Example: Curve comparison with/without separately specified derivatives}

\subsubsection{Chapman-Jouget waves}
The EOS structure is quite general, enabling many different computations, including basic models for reactive flows. Chapman-Jouget waves (detonation and deflagration) can be computed using a typical EOS for the products with only an initial state for the reactants. 
To begin, we assume that the reaction occurs instantaneously, and can be modeled as a discontinuity (much like a shock). This implies that the state of the products will lie on the Hugoniot curve that 
A Chapman-Jouget wave is determined using the same equations as the shock Hugoniot, except that the energy equation is written in terms of two distinct equations of state for the reactants and the products: 

% MathType!MTEF!2!1!+-
% faaagCart1ev2aaaKnaaaaWenf2ys9wBH5garuWqH1MyYLwyaeXatL
% xBI9gBaerbd9wDYLwzYbqee0evGueE0jxyaibaieYlf9irVeeu0dXd
% h9vqqj-hHeeu0xXdbba9frFj0-OqFfea0dXdd9vqaq-JfrVkFHe9pg
% ea0dXdar-Jb9hs0dXdbPYxe9vr0-vr0-vqpi0dc9GqpWqaaeaabiGa
% ciaacaqabeaadaqaaqaaaOqaaiaadwgacqGHRaWkcaWGWbWaaWbaaS
% qabeaadaqadaqaaiaadchaaiaawIcacaGLPaaaaaGcdaqadaqaaiaa
% dAhacaGGSaGaamyzaaGaayjkaiaawMcaaiaadAhacqGHRaWkdaWcaa
% qaaiaaigdaaeaacaaIYaaaamaabmaabaGaamyDamaaBaaaleaacaWG
% ZbaabeaakiabgkHiTiaadwhaaiaawIcacaGLPaaadaahaaWcbeqaai
% aaikdaaaGccqGH9aqpcaWGLbWaaSbaaSqaaiaaicdaaeqaaOGaey4k
% aSIaamiCamaaCaaaleqabaWaaeWaaeaacaWGYbaacaGLOaGaayzkaa
% aaaOWaaeWaaeaacaWG2bWaaSbaaSqaaiaaicdaaeqaaOGaaiilaiaa
% dwgadaWgaaWcbaGaaGimaaqabaaakiaawIcacaGLPaaacaWG2bWaaS
% baaSqaaiaaicdaaeqaaOGaey4kaSYaaSaaaeaacaaIXaaabaGaaGOm
% aaaacaWG1bWaaSbaaSqaaiaadohaaeqaaaaa!5627!
\[e + {p^{\left( p \right)}}\left( {v,e} \right)v + \frac{1}{2}{\left( {{u_s} - u} \right)^2} = {e_0} + {p^{\left( r \right)}}\left( {{v_0},{e_0}} \right){v_0} + \frac{1}{2}{u_s}\]

As before, we may derive the Hugoniot relation, though this time the ``before" and ``after" states are the upstream reactants and the downstream products, respectively:
% MathType!MTEF!2!1!+-
% faaagCart1ev2aaaKnaaaaWenf2ys9wBH5garuWqH1MyYLwyaeXatL
% xBI9gBaerbd9wDYLwzYbqee0evGueE0jxyaibaieYlf9irVeeu0dXd
% h9vqqj-hHeeu0xXdbba9frFj0-OqFfea0dXdd9vqaq-JfrVkFHe9pg
% ea0dXdar-Jb9hs0dXdbPYxe9vr0-vr0-vqpi0dc9GqpWqaaeaabiGa
% ciaacaqabeaadaqaaqaaaOqaaiabfs5aejaadwgacqGHRaWkceWGWb
% GbaebacqqHuoarcaWG2bGaeyypa0JaaGimaaaa!3726!
\[\Delta e + \bar p\Delta v = 0\]

From conservation of mass and momentum, one may derive the relation: 

% MathType!MTEF!2!1!+-
% faaagCart1ev2aaaKnaaaaWenf2ys9wBH5garuWqH1MyYLwyaeXatL
% xBI9gBaerbd9wDYLwzYbqee0evGueE0jxyaibaieYlf9irVeeu0dXd
% h9vqqj-hHeeu0xXdbba9frFj0-OqFfea0dXdd9vqaq-JfrVkFHe9pg
% ea0dXdar-Jb9hs0dXdbPYxe9vr0-vr0-vqpi0dc9GqpWqaaeaabiGa
% ciaacaqabeaadaqaaqaaaOqaaiabeg8aYnaaBaaaleaacaaIWaaabe
% aakmaaCaaaleqabaGaaGOmaaaakiaadwhadaWgaaWcbaGaam4Caaqa
% baGcdaahaaWcbeqaaiaaikdaaaGccqGH9aqpcqaHbpGCdaahaaWcbe
% qaaiaaikdaaaGccaWG1bWaaSbaaSqaaiaadohaaeqaaOWaaWbaaSqa
% beaacaaIYaaaaOGaeyypa0JaeyOeI0YaaSaaaeaacqqHuoarcaWGWb
% aabaGaeuiLdqKaamODaaaaaaa!4317!
\[{\rho _0}^2{u_s}^2 = {\rho ^2}{u_s}^2 =  - \frac{{\Delta p}}{{\Delta v}}\]

This linear relation is called the Rayleigh line. 

Because of the different equations of state, the initial state of the reactants will lie below the Hugoniot curve for the products, and must be connected to this curve by the Rayleigh line. This implies that the changes in pressure and volume across the wave must have opposite signs, which breaks the Hugoniot into two branches\cite{RN1011}, corresponding to a compressive detonation wave and an expansive deflagration wave (expansive waves are not precluded by the second law of thermodynamics, as they are with shocks). In a Chapman-Jouget detonation/deflagration, the final state on the Hugoniot is uniquely determined by the point on the appropriate branch where the Rayleigh line is tangent to the Hugoniot. See Fig. 76 in Courant\cite{RN1011}, p. 210. Therefore, the wave can be uniquely determined with only the products EOS and an initial state, by numerically solving for that point on the appropriate branch.  

\subsubsection{Impedance Matching}

% Detonation constructor defined in src/Eos/Wave.h
% Another function defined in src/Eos/EOS.h
% You construct the class first (with an EOS)
% You then initialize the class with an initial state (wavestate = v0, e0, u0; plus p0) defined in EOS.h.
% The p0 is required because this is not a shock. If it were, then p0 would equal p(v0, e0)
% The value for e0 is typically chosen to be the natural initial energy plus the heat release from the chemical reaction, while p0 is left to be the natural initial pressure. 
% Deflagration works the exact same way, but with the other branch of the Hugoniot.
% Deflagration also may run into domain issues if you're not careful. 

\section{ExtEOS (abstract base class for extended EOS model)}

\begin{itemize}
	\item Zref (reference state where you know the internal state variables)
	\item Frozen vs. equilibrium
		\begin{itemize}
			\item Frozen freezes the IDOF at Zref
			\item Equilibrium immediately assumes the IDOFs are at their equilibrium state
		\end{itemize}
\end{itemize}

\begin{itemize}
	\item Reacting flows
		\begin{itemize}
			\item ZND Wave
		\end{itemize}
	\item Elastic solids
		\begin{itemize}	
			\item ElasticPlastic
		\end{itemize}
\end{itemize}

\section{Library of specific EOS models}

\begin{tabular}[]{| l |}
\hline
Ideal Gas\\
Stiffened Gas\\
JWL\cite{RN746}\\
Hayes\cite{RN837}\\
HayesBM\\
GenHayes\\
Davis\cite{RN743}\\
Mie-Gruneisen\cite{RN824}\\
Porous\cite{RN837}\\
PTequilibrium\\
\hline
\end{tabular}

\begin{itemize}
	\item Ideal Gas
	\item Stiffened Gas
	\item Hayes
	\item HayesBM
	\item Generalized Hayes (Completed Mie-Gruneisen)
	\item Mie-Gruneisen
	\item Davis
	\item JWL 
	\item Porous
	\item PTequilibrium
\end{itemize}

\begin{itemize}
	\item ArrheniusHE
	\item Elastic1D
	\item ElasticPlastic
		\begin{itemize}
			\item IDOFs track the deviatoric stress while EOS tracks the volumetric stress as well as plastic strain.
		\end{itemize}
	\item HEburn
\end{itemize}


\bibliographystyle{asmems4}
\bibliography{endnote}

%%%%%%%%%%%%%%%%%%%%%%%%%%%%%%%%%%%%%%%%%%%%%%%%%%%%%%%%%%%%%%%%%%%%%%
\appendix       %%% starting appendix
\section*{Appendix A: Head of First Appendix}
Compile command: \verb!g++ -L $PATH_TO_EOSLIB_LIBRARIES -I $PATH_TO_EOSLIB_INCLUDES -lMaterials -lEOS LowLevelHugoniot.C!

%%%%%%%%%%%%%%%%%%%%%%%%%%%%%%%%%%%%%%%%%%%%%%%%%%%%%%%%%%%%%%%%%%%%%%
\section*{Appendix B: Head of Second Appendix}
\subsection*{Subsection head in appendix}
The equation counter is not reset in an appendix and the numbers will
follow one continual sequence from the beginning of the article to the very end as shown in the following example.
\begin{equation}
a = b + c.
\end{equation}

\end{document}

